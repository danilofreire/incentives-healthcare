\documentclass[a4paper, reqno, 12pt]{article}

% Packages
\usepackage[brazilian,english]{babel}
\usepackage[T1]{fontenc}
\usepackage[utf8]{inputenc}
\usepackage[super]{nth}
\usepackage{color, graphics, graphicx, amsmath, times, amsfonts, amssymb}
\usepackage{listings,  palatino, fancyhdr, setspace, bbm, type1cm, titletoc, dsfont}
\usepackage{amsthm, underscore, lscape, dcolumn, arydshln}
\usepackage[usenames,svgnames,dvipsnames]{xcolor}
\usepackage[sharp]{easylist}
\usepackage[a4paper,top=1in,bottom=1in,left=1in,right=1in]{geometry}
\usepackage[font=small,format=plain,labelfont=bf,up,textfont=it,up]{caption}
\usepackage[pdftex,plainpages=false,pdfpagelabels,pagebackref = false,colorlinks=true,citecolor=DarkBlue,linkcolor=NavyBlue,urlcolor=DarkRed,filecolor=DarkBlue,bookmarksopen=true]{hyperref}
\usepackage{tabularx}
\usepackage[para,online,flushleft]{threeparttable}  
\usepackage{booktabs}
\usepackage{multirow}
\usepackage{bm}
\usepackage{natbib}
\usepackage{subcaption, rotating}
\usepackage{pgf, tikz, pgfplots, mathrsfs}

\usetikzlibrary{positioning,arrows,calc,fit}

\pgfplotsset{compat=newest}

\definecolor{wrwrwr}{rgb}{0.38,0.38,0.38}
\definecolor{rvwvcq}{rgb}{0.08,0.38,0.75}

\tikzset{
block/.style={
  draw, 
  rectangle, 
  minimum height=1.5cm, 
  minimum width=3cm, align=center
  }, 
line/.style={->,>=latex'}
}

\tikzset{container/.style={draw, rectangle, dashed, inner sep=1.5em}}

\bibpunct{(}{)}{;}{a}{}{,}

\pagestyle{plain}
\renewcommand{\arraystretch}{1.2}
\sloppy

%\renewcommand{\thepage}{}
\renewcommand{\bibname}{References}
\newcolumntype{Y}{>{\raggedleft\arraybackslash}X}
\newcolumntype{R}{>{\raggedleft\arraybackslash}X}

\newcommand{\wh}[1] {\widehat { #1 }}

\newcommand{\twopd}[4]{\left\{\begin{array}{ll} #1 & \text{if } #2 \\ #3 & \text{if } #4 \end{array} \right.}

\newcommand{\twopdo}[3]{\left\{\begin{array}{ll} #1 & \text{if } #2 \\ #3 & \text{otherwise} \end{array} \right.}

\newcommand{\threepd}[6]{ \left\{ \begin{array}{lll} #1 & \text{if } #2 \\ 
#3 & \text{if } #4 \\ #5 & \text{if } #6 \end{array} \right.}

\newcommand{\threepdo}[5]{ \left\{ \begin{array}{lll} #1 & \text{if } #2 \\ 
#3 & \text{if } #4 \\ #5 & \text{otherwise} \end{array} \right.}

% Bold letters
\newcommand{\X}{\mathbf{X}}
\newcommand{\Y}{\mathbf{Y}}
\newcommand{\Z}{\mathbf{Z}}
\newcommand{\E}{\mathbb{E}}
\newcommand{\I}{\mathbf{I}}
\newcommand{\m}{\mathbf{m}}
\newcommand{\f}{\mathbf{f}}
\newcommand{\e}{\mathbf{e}}
\newcommand{\V}{\mathbf{V}}
\newcommand{\W}{\mathbf{W}}
\newcommand{\R}{\mathbf{R}}
\newcommand{\A}{\mathbf{A}}
\newcommand{\B}{\mathbf{B}}
\newcommand{\D}{\mathbf{D}}
\newcommand{\T}{\mathbf{T}}
\newcommand{\F}{\mathbf{F}}
\renewcommand{\O}{\mathbf{O}}
\renewcommand{\H}{\mathbf{H}}
\newcommand{\g}{\mathbf{g}}
\renewcommand{\r}{\mathbf{r}}
\newcommand{\s}{\mathbf{s}}
\newcommand{\dd}{\mathbf{d}}
\newcommand{\uu}{\mathbf{u}}
\newcommand{\w}{\mathbf{w}}
\newcommand{\x}{\mathbf{x}}
\newcommand{\vv}{\mathbf{v}}
\newcommand{\y}{\mathbf{y}}
\newcommand{\z}{\mathbf{z}}
\newcommand{\Beta}{\boldsymbol{\beta}}
\newcommand{\bveps}{\boldsymbol{\varepsilon}}
\newcommand{\bvphi}{\boldsymbol{\varphi}}
\newcommand{\btheta}{\boldsymbol{\theta}}
\newcommand{\bups}{\boldsymbol{\upsilon}}
\newcommand{\bgamma}{\boldsymbol{\gamma}}
\newcommand{\bpi}{\boldsymbol{\pi}}
\newcommand{\arrowp}{\stackrel{p}{\rightarrow}}
\newcommand{\0}{\mathbf{0}}
\newcommand{\Prob}{\mathbb{P}}
\newcommand{\nn}{\nonumber}
\newcommand{\indf}{\mathds{1}}

% Theorems
\newtheorem{theorem}{Theorem}
\newtheorem{lemma}{Lemma}
\newtheorem{claim}{Claim}
\newtheorem{axiom}[theorem]{Axiom}
\newtheorem{definition}{Definition}
\newtheorem{corollary}{Corollary}
\newtheorem{proposition}{Proposition}
\newtheorem{assumption}{Assumption}
%\newenvironment{proof}[1][Proof]{\textbf{#1.} }{\ \rule{0.5em}{0.5em}}

% Illustration
\theoremstyle{definition}
\newtheorem{example}{Example}%[subsection]
\newtheorem{hypothesis}{Hypothesis}%[subsection]

% Claims in the theorems (without numbering)
\theoremstyle{remark}
\newtheorem*{claimnn}{Claim}

% Title
\title{Supplementary Materials for Incentives for Preventive Healthcare Provision%\thanks{I would like to thank Livio DiLonardo, Guilherme Fasolin, Saad Gulzar, Cathy Hafer, Matheus Hardt, Carlo Holz, Eduardo Mello, Dimitri Landa, Guisela Pereira, Adam Przeworski, Matias Spektor, Denis Stukal, Marcos Tourinho, and Scott Tyson for their valuable comments. I also thank participants at the MPSA, NYU PE Seminar, and SPSA for their suggestions. All the remaining errors are entirely my responsibility.}
}
\author{Danilo Freire \and Umberto Mignozzetti
}

\date{}

\begin{document}
\maketitle

\doublespacing

\appendix

\tableofcontents

\doublespacing

\section{A Formal Model of the Incentives for Preventive Healthcare Provision in Teams}

XXXX

\subsection{Primitives of the Model}

XXXX

\subsection{No Performance Bonus}

XXXX

\subsection{Individual Performance Bonus}

XXXX

\subsection{Team Performance Bonus}

XXXX

\subsection{Comparative Statics}

XXXX

\subsection{Paper Example}

The Wolfram Alpha code for the paper example is the following:

\begin{verbatim}
RNoI[x_] := 0*x

RInd[x_] := 2 (x^(1/2))*(1 - x)

RCol[x_] := (2 (x^(1/2))*(1 - 2 x))/(1 - x)^2

Plot[{RNoI[x], RInd[x], RCol[x]}, {x, 0, 0.5}, 
  PlotLegends -> {"No Incentives", "Individual Incentives", 
    "Peer Incentives"}, 
  PlotLabel -> "Mechanisms' Revenues", 
  PlotStyle -> {Thickness[0.015], Thickness[0.01],
    Thickness[0.01]}, 
  AxesLabel -> {"Compensation", "Revenue"}]
\end{verbatim}

\section{APSA Experimental Section Standard Report}

\subsection{Question addressed by the experiment}

In this experiment, we study the impact of monetary incentives on the performance of health care workers in Brazil. The study aims to find which financial incentive scheme is better to improve the performance of the workers in the field.

Most of the developmental economics papers on incentives for public service providers highlight that better compensation works. However, these papers focus on services that are individually provided, such as a teacher in a classroom. In health care, rarely the efforts are undertaken independently, and this poses the question: what is the best way to improve public health care provision, when teams, rather than individuals, provide the service?

In this paper, we undertake the first effort to bridge this gap by testing whether team incentives are more effective than individual incentives to improve preventive healthcare provision in Brazil.

The Brazilian case is well suited for this test for two main reasons. First, disease borne by the Aedes aegypti is one of the leading causes of congenital disabilities, hospitalization, and casualties in the global south. Second, a pair of health care workers are responsible for visiting households and exterminate the disease in Brazil. Therefore, we could test individual versus team bonuses in a setting that is relevant to welfare.

\subsection{Hypothesis tested}

We tested which types of bonuses are better for improving the performance of health care workers in Brazil. We tested three incentive schemes: first, monetary compensation with no performance bonuses; second, individual compensation for performance above the median in the headquarter they were assigned; and third, peer bonus compensation for team performance above the median in the headquarters. Theoretically, team bonuses should be more effective, as it increases peer monitoring, diminishing the free-riding problem. However, if there are fewer complementarities, individual rewards can be more productive. This poses the question: which incentive scheme is better for community-based prevention of Aedes aegypti borne diseases in Brazil?

\section{Subjects and Context}

\subsection{Eligibility and exclusion criteria}

\subsubsection{Why was this subject pool selected?}

The subjects participating in the experiment were selected based on a Facebook recruitment Ad posted in the Municipality of Rio Verde, Goiás State, Brazil. We chose this town because we had an agreement with the mayor's office, and also because it had a considerable spike in dengue fever cases in 2018. One of the Facebook ads follows below.

\pagebreak

\begin{figure}
\centering
\includegraphics[width=\textwidth]{figures/AppCrecruitment1.png}
\caption{Facebook Recruitment Ad with Pilot Participants' picture.\label{appcfbad}}
\end{figure}

The Facebook Ad redirected the participant to a Google Forms where s/he filled up a participation eligibility form. The Google form used is below.

\pagebreak

\begin{figure}
\centering
\includegraphics[width=0.8\textwidth]{figures/AppCgoogleformsrecruiting.png}
\caption{Google Form Recruitment.\label{appcgform}}
\end{figure}

After that, we send them an email, asking questions on demographics and other characteristics.

\subsubsection{Who was eligible to participate in the study?}

Adults, 18 old or older.

\subsubsection{What would result in the exclusion of a participant?}

Not showing up during the intervention date.

\subsubsection{Were any aspects of recruitment changed after the recruitment began?}

No.

\subsubsection{Procedures used to recruit and select participants}

We posted a Facebook ad in the municipality participating in the intervention. The ad invited potential participants to join the experiment and explained the compensation. If they clicked the ad, they were directed to Google Forms, where they needed to fill up their e-mails and personal information.

\subsubsection{Recruitment dates}

The online recruitment took place right after filling up the form. We send potential participants information regarding Aedes aegypti prevention and a link for a pre-treatment survey. After filled, participants were informed about the live training dates and selected preferred times to participate. They then had to show up for the two-hour training session.

The training date for the pilot was April 13, 2018. The training date for the intervention was May 4, 2018. A picture of the training day before the field follows below.

\pagebreak

\begin{figure}
\centering
\includegraphics[width=\textwidth]{figures/AppCtrainingdaypic.png}
\caption{Training Day.\label{appctrday}}
\end{figure}

\subsubsection{Settings and locations where the data were collected}

There are two primary datasets used in the paper. First, we collected productivity data from the cell phones used by the participants in the field. Second, we collected governmental data on diseases transmitted by the Aedes aegypti from the Brazilian Ministry of Health diseases notification system (SINAN).

The first dataset is generated by the participants pictures in the field. The second dataset is comprised of governmental records of people that checked into hospitals and were confirmed having dengue fever. This data has the address of the person, that we then used to geolocate the case. Among the diseases transmitted by the Aedes aegypti, dengue fever is the most frequent and easy to detect.

\subsubsection{Relevant specifics about the population}

All the participants were from the municipality of Rio Verde, in the state of Goiás, in Brazil. A map of the municipality follows below:

\begin{figure}
\centering
\includegraphics[width=\textwidth]{figures/AppCmap1.png}
\caption{Rio Verde, GO, Brazil Map.\label{appcmaprv}}
\end{figure}

\pagebreak

\section{Allocation Methods}

\subsection{Randomization procedure}

We generated the random assignment using block randomization applied on seven variables:

\begin{enumerate}
    \item Above median altruism
    \item Age
    \item Above median religiosity score
    \item Above median Political Engagement
    \item Above median social engagement
    \item Above two minimum wage family income
    \item Above median Facebook popularity (number of friends)
\end{enumerate}

These variables were computed from surveys that we collected prior to the intervention. To test the balance, we use F-tests on the block randomization data. The balance tests follow below.

\pagebreak

\begin{table}[!htbp] \centering 
  \caption{Covariate Balance on Individual Characteristics} 
  \label{appctabcovbalindv} 
\begin{tabular}{@{\extracolsep{5pt}} lccccc}
\\[-1.8ex]\hline 
\hline \\[-1.8ex] 
Variable & Control & Individual & Collective & F-Statistic & P-Value \\
\hline \\[-1.8ex]
Altruist &  0.63 &  0.68 &  0.62 & 0.239 & 0.788 \\ 
Age & 25.01 & 26.00 & 26.90 & 0.754 & 0.472 \\ 
Religiosity &  0.53 &  0.38 &  0.45 & 1.487 & 0.228 \\ 
Pol. Engagement &  0.69 &  0.65 &  0.68 & 0.163 & 0.849 \\ 
Soc. Engagement &  0.59 &  0.62 &  0.67 & 0.454 & 0.636 \\ 
Above 2 Min. Wage &  0.34 &  0.35 &  0.29 & 0.336 & 0.715 \\ 
FB Popularity &  0.53 &  0.56 &  0.58 & 0.175 & 0.840 \\ 
\hline\hline \\[-1.8ex] 
\end{tabular}
\end{table}

The treatment is balanced among the participants' covariates. For the territorial assignment, we have the following pre-treatment unweighted results:

\pagebreak

\begin{table}[!htbp] \centering 
  \caption{Covariate Balance Un-Weighted Territorial Assignment} 
  \label{appcunweightbalter} 
\begin{tabular}{@{\extracolsep{5pt}} lccccc}
\\[-1.8ex]\hline 
\hline \\[-1.8ex] 
Variable & Control & Individual & Collective & F-Statistic & P-Value \\
\hline \\[-1.8ex] 
Number of Houses & 276.41 & 243.58 & 255.20 &  2.368 & 0.098 \\ 
Number of Households & 885.67 & 750.34 & 785.18 &  3.765 & 0.026 \\ 
Avg. Household Size &   3.21 &   3.07 &   3.07 &  6.566 & 0.002 \\ 
Log Avg. Income &   6.47 &   6.86 &   6.85 &  9.213 & 0.000 \\ 
Cases Before Treatment &  13.78 &   5.94 &   4.66 & 15.638 & 0.000 \\
\hline\hline \\[-1.8ex] 
\end{tabular}
\end{table}

As the territorial assignment is unbalanced, I reweighted the assignment using the census sector information. To reweight I used propensity score matching. After reweight, we have the following results.

\pagebreak

\begin{table}[!htbp] \centering 
  \caption{Covariate Balance Weighted Territorial Assignment} 
  \label{appcweightbalter} 
\begin{tabular}{@{\extracolsep{5pt}} lccccc} 
\\[-1.8ex]\hline 
\hline \\[-1.8ex] 
Variable & No Bonus & Individual & Collective & F-Statistic & P-Value \\
\hline \\[-1.8ex] 
Number of Houses & 276.41 & 282.04 & 281.19 & 0.068 & 0.935 \\ 
Number of Households & 885.67 & 903.04 & 900.39 & 0.060 & 0.942 \\ 
Avg. Household Size &   3.21 &   3.21 &   3.20 & 0.008 & 0.992 \\ 
Log Avg. Income &   6.47 &   6.45 &   6.45 & 0.084 & 0.920 \\ 
Cases Before Treatment &  13.78 &  10.24 &  10.77 & 1.443 & 0.240 \\
\hline\hline \\[-1.8ex] 
\end{tabular}
\end{table}

The reweighting successfully balance the covariates around the territorial assignment.

\subsection{Random assignment}

\subsubsection{Units of randomization}

I randomized the treatment at the level of the participant.

\subsubsection{Cluster random assignment}

No cluster random assignment was used. However, in the differences-in-differences estimation, we use the census sector where the pair worked to cluster the standard errors.

\subsection{Blinding}

\subsubsection{Were participants unaware of the treatment assignment?}

Yes. They knew their incentive schemes, but not other incentive schemes. To ensure that there would be no spillovers, we divided each treatment assigned teams into three different headquarters:

\begin{enumerate}
    \item Assembleia de Deus Church (No bonus)
    \item Videiras Church (Individual reward)
    \item Sara Nossa Terra Church (Collective bonus)
\end{enumerate}

These headquarters were away from each other to avoid contact between participants.

\subsubsection{Were those administering the intervention unaware of the random assignment?}

No. The head research assistants in each group knew that different treatment would be administered in other research headquarters. This measure was taken in case the treatment spillover, and we needed to remove pairs that were disturbing the work from the field.

\subsubsection{Checked whether blind was successful?}

The single-blind was successful. We received no complaints during the intervention day. After the intervention day, participants became aware of the experiment. The post-treatment spillover made the post-treatment survey ineffective because they refused to cooperate with the researchers.

\section{Treatments}

\subsection{Descriptions of the intervention}

\subsubsection{Treatment groups}

We have three treatment groups:

\begin{enumerate}
    \item Control (no performance bonus)
    \item Individual treatment
    \item Collective treatment
\end{enumerate}

In the control group, a bonus was assigned without performance concerns. 

In the individual treatment group, we explained the participants that we would rank the performance based on number of visited houses, number of breeding sites removed and cleaned, and number of larvae exterminated. Participants would have their performance ranked and the ones above the median would double their monetary reward (from BRL 110.00 to BRL 220.00). 

In the collective treatment group, we explained the participants that we would rank their team performance based on number of visited houses, number of breeding sites removed and cleaned, and number of larvae exterminated. We would sum their individual performance with the performance of their field peers, and then rank the teams. Teams with performance above the median would double their monetary reward (from BRL 110.00 to BRL 220.00, for each person in the team).

The fieldwork was performed in pairs, to mimic the Aedes Aegypti actual practices in Brazil.

\subsubsection{Control groups}

The control group received a monetary reward with no performance requirement.

\subsubsection{Experimental instructions}

In the study day, around 5AM, we randomly assigned pairs, and send emails to the individuals directing them to each of the headquarters, based on their treatment status.

We had three study headquarters: the control headquarters (Assembleia de Deus church); the individual treatment headquarters (Videiras church); and the peer treatment headquarters (Sara Nossa Terra church). We printed set of leaflets with places that they had to administer the treatment. One example of a leaflet follows below:

\pagebreak

\begin{figure}
\centering
\includegraphics[width=0.8\textwidth]{figures/AppCsarabase1.png}
\caption{Example Working Block.\label{appcevwbl}}
\end{figure}

We handed in the cellphones and the working kits for them. An example of the working kits design follows below:

\pagebreak

\begin{figure}
\centering
\includegraphics[width=0.8\textwidth]{figures/AppCteamsclothes.jpg}
\caption{Teams Uniforms.\label{appcteamunif}}
\end{figure}

The cellphone was a Samsung, model Galaxy A1. A picture with a testing of the data collection app follows below.

\pagebreak

\begin{figure}
\centering
\includegraphics[width=0.4\textwidth]{figures/AppCcellphone.jpeg}
\caption{Cellphone with data collection app in test.\label{appccpdcit}}
\end{figure}

The informative leaflet distributed to households follow here:

\pagebreak

\begin{figure}
\includegraphics[width=0.8\textwidth]{figures/AppCFolder1.jpg}
\includegraphics[width=0.8\textwidth]{figures/AppCFolder2.jpg}
\caption{Example leaflet handed to households.\label{appcexhhleafl}}
\end{figure}

Then, at around 11:00am we administered the main treatments. They consisted in tell the following instructions to the participants, when we handed the leaflet with their working places:

\begin{itemize}
    \item \textbf{Control:} we told them that there would be an increase in their payment without explaining the reason. We told them about how performance would be measured, without saying anything about compensation based on performance.
    \item \textbf{Individual treatment:} we told the participants that we would measure their performances in the field, and that we would rank all the performances. The individuals with performance above the median would receive the bonus. We also explained in lay terms what was a median, and how we would measure performance.
    \item \textbf{Peer treatment:} we told the participants that we would measure their performances in the field, and that we would rank all the performances. The pairs with performance above the median would receive the bonus. We also explained in lay terms what was a median, and how we would measure performance.
\end{itemize}

\subsection{How and when manipulations were administered}

\subsubsection{Method of delivery}

The manipulations were delivered in person, by the chief headquarter RA, first to all the pairs in a speech, then to each pair in person. For each pair, we reinforced the instructions while handing in the leaflets.

\subsubsection{Software used to administer the treatment}

The randomization was performed in R. The treatment administration was done in person. 

\section{Results}

\subsection{Outcome measures and covariates}

\subsubsection{Outcome measures}

There are two groups of outcomes. First, the performance outcomes that are collected from the cellphone used by each participant in the field. Second, the governmental data reported by the health care facilities in the municipality, with all dengue cases.

For the performance data we have:

\begin{enumerate}
    \item Geolocated pictures of house fronts as a proof that they visited to house.
    \item Pictures of exterminated breeding sites.
    \item Video of exterminated a. aegypti larvae.
\end{enumerate}

For governmental data, we have all the people infected by dengue fever that went to the hospital. There were no reported cases of Zika and Chikungunya, as the detection of these diseases are less standardized that Dengue. Moreover, the mayors office have no recollection of microcephaly incidence in the area during the period.

The data then corresponds to all dengue cases, with address of the infected person, date of check-in to the hospital, and the severity of the case.

\subsubsection{Covariates}

We collected information on demographics.

\subsubsection{Which outcomes and subgroup analysis were specified prior to the experiment?}

As the mayors office did not handed in the infections data prior to the intervention, we asked the neighborhoods that had higher dengue fever incidence prior to the intervention, and placed the territorial assignment in these places. However, due to absence of micro-level infectious disease data, we could not consider them prior to the intervention.

\subsubsection{Exploratory analysis}

We run a pilot with 20 pairs before main intervention. We did not administered the incentivized intervention in the pilot to avoid contamination in the main results. The pilot helped us to update the expectations regarding the productivity in the field. Previously, we projected that we should assign around 20 houses per pairs. However, in the pilot we learned that pairs could visit from 40 to 60 houses per pair. In around one-thirty of them nobody answered the door, and in most of them the yards were too small to find any breeding sites. Moreover, most of the places with dengue in the previous week were visited by the municipal health care workers.

\subsection{CONSORT}

\subsubsection{Number of subjects initially assessed for eligibility}

558 individuals filled in the Google Form. 256 showed for the training session one day before the intervention. 205 individuals showed for the intervention. 197 remained until the end of the intervention.

\subsubsection{Exclusions prior to random assignment}

From all eligible people, we excluded the ones that missed the training session. This accounts for 45.8\% of the total.

\subsubsection{Subjects initially assigned to each experimental group}

\begin{enumerate}
    \item 68 to control group.
    \item 68 to individual treatment group.
    \item 69 to peer treatment group.
\end{enumerate}

\subsubsection{Proportion received x not received intervention}

\begin{enumerate}
    \item 58 received the control intervention out of 68 in the control group.
    \item 64 received the intervention out of 68 in the individual treatment group.
    \item 75 received the intervention out of 69 in the peer treatment group.
\end{enumerate}

\subsubsection{Why did not receive intervention?}

The individuals did not received the intervention if they missed the field. They were then excluded from the data.

\subsubsection{Number subjects each group dropped experiment}

No subject were dropped from the assignment. Six subjects joined the experiment in the intervention day. We assigned them to the peer intervention group.

\section{Statistical analysis}

\subsection{Describe statistical analysis}

We run four types of statistical analysis:

\begin{enumerate}
    \item Balance tests
    \item Performance outcome analysis
    \item Disease incidence analysis
    \item Differences-in-differences analysis
\end{enumerate}

The main equations for the performance outcomes are:

\[
Y_i = \alpha + \beta D_i + \varepsilon_i
\]

Where $\beta$ is the outcome of interest, $Y_i$ is the outcome for case $i$, and $D_i$ is the treatment status for case $i$. And the code is:

\begin{verbatim}
mod <- lm(outc ~ treat, data = dataframe)
\end{verbatim}

The main equations for the disease incidence are:

\[
Y_i = \alpha + \beta D_i + \varepsilon_i
\]

Where $\beta$ is the outcome of interest, $Y_i$ is the outcome for case $i$, and $D_i$ is the treatment status for case $i$. The only difference is that the estimates are weighted using the propensity score matching. And the code is:

\begin{verbatim}
mod <- lm(outc ~ treat, 
          data = dataframe, 
          weights = dataframe$weights)
\end{verbatim}

Finally, the differences-in-differences estimator has the following form:

\[
Y_i = \alpha + \gamma Time_i + \theta Treat_i + \beta Time_i \times Treat_i + \varepsilon_i
\]

And our quantity of interest is $\beta$. The codes for the regression models I run are as follows:

\begin{verbatim}
mod <- felm(outc ~ time+treat+DID|0|0|CensusSectors, 
            data = dataframe)
\end{verbatim}

This regression uses the package \texttt{lfe}, as it facilitates the use of cluster-robust standard errors.

Table \ref{appcprodregs} shows the performance in the field estimation.

\pagebreak

\begin{table}[!htbp] \centering 
  \caption{Field Productivity} 
  \label{appcprodregs} 
\begin{tabular}{@{\extracolsep{5pt}}lcccc} 
\\[-1.8ex]\hline 
\hline \\[-1.8ex] 
 & \multicolumn{4}{c}{\textit{Dependent variable:}} \\ 
\cline{2-5} 
\\[-1.8ex] & Houses & Houses & Breeding & Larvae \\ 
& Visited & Visited ($<$ 2min.) & Sites & Exterminated \\ 
\hline \\[-1.8ex] 
 Individual Bonus & $-$9.879$^{**}$ & $-$8.044$^{**}$ & 25.118$^{***}$ & 0.022 \\ 
  & (4.906) & (4.018) & (5.251) & (0.058) \\ 
  & & & & \\ 
 Collective Bonus & $-$7.554$^{*}$ & $-$6.479$^{*}$ & 21.512$^{***}$ & 0.163$^{**}$ \\ 
  & (4.428) & (3.571) & (4.049) & (0.076) \\ 
  & & & & \\ 
\hline \\[-1.8ex] 
Observations & 197 & 197 & 197 & 197 \\ 
Residual SE & 27.301 & 22.153 & 30.545 & 0.425 \\ 
\hline 
\hline \\[-1.8ex] 
\textit{Note:}  & \multicolumn{4}{r}{Robust Standard Errors in Parenthesis. $^{*}$p$<$0.1; $^{**}$p$<$0.05; $^{***}$p$<$0.01} \\ 
\end{tabular} 
\end{table}

The performance regressions shows that:

\begin{enumerate}
    \item Performance bonus, regardless of the nature, lower the number of houses visited. This because to score high they needed to find breeding sites and larvae, not only visit houses.
    \item Performance bonus, regardless of the nature, increases the number of breeding sites cleaned up by the workers. The effect seems to be higher in the individual bonus group.
    \item Only collective bonus increase the chance of finding larvae. Note that the effort required to find larvae is considerably higher than the effort required to find breeding sites.
\end{enumerate}

For the disease incidence regressions, the results in Table \ref{appcregdiseaseinc1} show all null effect.

\pagebreak

\begin{table}[!htbp]
  \caption{Disease incidence in Rio Verde Census Sectors} \centering 
  \label{appcregdiseaseinc1} 
\begin{tabular}{@{\extracolsep{5pt}}lcccccc}
\\[-1.8ex]\hline 
\hline \\[-1.8ex] 
 & \multicolumn{6}{c}{\textit{Dependent variable:}} \\ 
\cline{2-7} 
\\[-1.8ex] & Before & 15 days & 30 days & 60 days & 90 days & 30-60 days \\ 
& Intv. & After Intv. & After Intv. & After Intv. & After Intv. & After Intv. \\
\hline \\[-1.8ex] 
 Individual & $-$3.534 & 0.116 & 0.551 & 1.551 & 1.534 & 0.936 \\ 
  & (2.919) & (1.338) & (2.193) & (3.260) & (3.258) & (1.188) \\ 
  & & & & & & \\ 
 Collective & $-$3.008 & $-$1.329 & $-$0.826 & 0.344 & 1.091 & 1.068 \\ 
  & (4.174) & (1.440) & (1.971) & (3.233) & (3.610) & (1.582) \\ 
  & & & & & & \\ 
\hline \\[-1.8ex] 
Observations & 139 & 139 & 139 & 139 & 139 & 139 \\ 
Residual Std. Error & 9.442 & 4.076 & 6.095 & 8.896 & 9.341 & 3.536 \\ 
\hline 
\hline \\[-1.8ex] 
\textit{Note:}  & \multicolumn{6}{r}{Robust Standard Error in Parenthesis. $^{*}$p$<$0.1; $^{**}$p$<$0.05; $^{***}$p$<$0.01} \\
\end{tabular} 
\end{table}

Finally, for the differences-in-differences models, the results are in Table \ref{appcdifindifdiseases}.

\pagebreak

\begin{table}[!htbp] \centering 
  \caption{Disease Incidence -- Differences in Differences Model} 
  \label{appcdifindifdiseases} 
\begin{tabular}{@{\extracolsep{5pt}}lccccc} 
\\[-1.8ex]\hline 
\hline \\[-1.8ex] 
 & \multicolumn{5}{c}{\textit{Dependent variable:}} \\ 
\cline{2-6} 
\\[-1.8ex] & N Infected & N Infected & N Infected & N Infected & N Infected \\
& (full) & (12 weeks) & (8 weeks) & (4 weeks) & (indv. x collect.) \\ 
\hline \\[-1.8ex] 
 DID & $-$0.103$^{*}$ & $-$0.130 & $-$0.167 & $-$0.097 & $-$0.029 \\ 
  & (0.053) & (0.091) & (0.118) & (0.166) & (0.035) \\ 
  & & & & & \\ 
\hline \\[-1.8ex] 
Observations & 7,452 & 3,312 & 1,656 & 828 & 2,201 \\ 
Residual SE & 1.004 & 1.319 & 1.442 & 1.513 & 0.381 \\ 
\hline 
\hline \\[-1.8ex] 
\textit{Note:} & \multicolumn{5}{r}{Robust standard error, clustered at the census-sector level, in parentheses.} \\ 
& \multicolumn{5}{r}{$^{*}$p$<$0.1; $^{**}$p$<$0.05; $^{***}$p$<$0.01} \\ 
\end{tabular} 
\end{table}

Table \ref{appcdifindifdiseases} shows the following:

\begin{enumerate}
    \item There is a 10.3\% effect on the full data model.
    \item The other windows are not significant but the sign is consistent with lower disease incidence.
    \item Comparing individual and collective, collective decreases by 2.9\%, but the result is not significant.
\end{enumerate}

\subsection{Standard errors}

The robust standard errors are computed using \texttt{vcovHC} function in the package \texttt{sandwich}. The standard error is the same as the Stata's \textit{comma-robust}.

\subsection{Attrition}

\subsubsection{Missing data}

The data we have is complete, as during the intervention no pair or individual participant withdrew from the research. Between the recruitment and the training we had a considerable number of dropouts. The numbers are:

\begin{itemize}
    \item \textbf{Signed the Google Form:} 558
    \item \textbf{Filled surveys:} 407
    \item \textbf{Showed for in-person training:} 256
    \item \textbf{Participated in the experiment:} 205
\end{itemize}

\subsubsection{Analyze pre-treatment variables to check reasons}

The only variables we have for all 558 individuals that filled up the Google Form is age and gender. Comparing them with the participants in the field shows that the attrition does not seem to have a detectable pattern.

\pagebreak

\begin{table}[!htbp] \centering 
  \caption{Attrition Analysis} 
\begin{tabular}{@{\extracolsep{5pt}} lcccccc} 
\\[-1.8ex]\hline 
\hline \\[-1.8ex] 
Variable & Attrition & No Bonus & Individual & Collective & F-Stat & P-Val \\ 
\hline \\[-1.8ex] 
Age &  26.17 & 25.04 & 26.07 & 26.00 & 0.461 & 0.710 \\ 
Female &   0.72 &  0.79 &  0.79 &  0.74 & 0.805 & 0.492 \\ 
Has Vehicle &   0.35 &  0.27 &  0.37 &  0.37 & 0.651 & 0.582 \\ 
N & 341 & 73 & 71 & 73 &  &  \\ 
\hline \\[-1.8ex] 
\end{tabular} 
\end{table} 

\begin{figure}
\centering
\includegraphics[width=0.8\textwidth]{figures/appccasesdengue.pdf}
\caption{Disease incidence before and after.\label{appcdisincbaf}}
\end{figure}

\pagebreak

\subsubsection{Method for addressing missing data}

The missing cases generated no outcomes nor participated in the research in any way. Therefore, we excluded them from the analysis.

\section{Other information}

\subsection{IRB}

We received IRB from the New York University and Fundação Getulio Vargas.

\begin{itemize}
    \item NYU IRB number: IRB-FY2017-17
    \item FGV IRB number: IRB-01/2017
\end{itemize}

\subsection{Pre-registration}

The study was pre-registered using the EGAP pre-registry tool, by the number 20180504AA.

\subsection{Funding}

The study was funded by the FGV applied research grant officer. They had no interference on the study results.

\subsection{Replication dataset}

The replication materials is in the following website: \url{https://github.com/umbertomig/dissertationNYU}.

\subsection{Thanks}

This research benefited from the contribution of the following research assistants:

\begin{itemize}
    \item Natalia Liberato
    \item Guilherme Fasolin
    \item Phillipe Guedon
    \item Vitor Sion
    \item Fatima Portella
    \item Giovanna França
    \item Catarina Roman
    \item Larissa Santos
    \item Leticia Santana
\end{itemize}

I am very thankful for all their help before, during, and after the field. All the remaining errors are my entire responsibility.

\subsection{Some field pictures}

\begin{figure}
\centering
\includegraphics[width=0.7\textwidth]{figures/AppCf12.png}
\caption{Field pictures}
\end{figure}

\pagebreak

\begin{figure}
\centering
\includegraphics[width=0.7\textwidth]{figures/AppCf14.png}
\caption{Field pictures}
\end{figure}

\pagebreak

\begin{figure}
\centering
\includegraphics[width=0.7\textwidth]{figures/AppCf15.png}
\caption{Field pictures}
\end{figure}

\clearpage

% Bibliography
\singlespacing
\begin{footnotesize}
\bibliographystyle{chicago}
\bibliography{biblio.bib}
\end{footnotesize}

\end{document}