\documentclass[a4paper, reqno, 12pt]{article}

% Packages
\usepackage[brazilian,english]{babel}
\usepackage[T1]{fontenc}
\usepackage[utf8]{inputenc}
\usepackage[super]{nth}
\usepackage{color, graphics, graphicx, amsmath, times, amsfonts, amssymb}
\usepackage{listings,  palatino, fancyhdr, setspace, bbm, type1cm, titletoc, dsfont}
\usepackage{amsthm, underscore, lscape, dcolumn, arydshln}
\usepackage[usenames,svgnames,dvipsnames]{xcolor}
\usepackage[sharp]{easylist}
\usepackage[a4paper,top=1in,bottom=1in,left=1in,right=1in]{geometry}
\usepackage[font=small,format=plain,labelfont=bf,up,textfont=it,up]{caption}
\usepackage[pdftex,plainpages=false,pdfpagelabels,pagebackref = false,colorlinks=true,citecolor=DarkBlue,linkcolor=NavyBlue,urlcolor=DarkRed,filecolor=DarkBlue,bookmarksopen=true]{hyperref}
\usepackage{tabularx}
\usepackage[para,online,flushleft]{threeparttable}  
\usepackage{booktabs}
\usepackage{multirow}
\usepackage{bm}
\usepackage{natbib}
\usepackage{subcaption, rotating}
\usepackage{pgf, tikz, pgfplots, mathrsfs}

\usetikzlibrary{positioning,arrows,calc,fit}

\pgfplotsset{compat=newest}

\definecolor{wrwrwr}{rgb}{0.38,0.38,0.38}
\definecolor{rvwvcq}{rgb}{0.08,0.38,0.75}

\tikzset{
block/.style={
  draw, 
  rectangle, 
  minimum height=1.5cm, 
  minimum width=3cm, align=center
  }, 
line/.style={->,>=latex'}
}

\tikzset{container/.style={draw, rectangle, dashed, inner sep=1.5em}}

\bibpunct{(}{)}{;}{a}{}{,}

\pagestyle{plain}
\renewcommand{\arraystretch}{1.2}
\sloppy

%\renewcommand{\thepage}{}
\renewcommand{\bibname}{References}
\newcolumntype{Y}{>{\raggedleft\arraybackslash}X}
\newcolumntype{R}{>{\raggedleft\arraybackslash}X}

\newcommand{\wh}[1] {\widehat { #1 }}

\newcommand{\twopd}[4]{\left\{\begin{array}{ll} #1 & \text{if } #2 \\ #3 & \text{if } #4 \end{array} \right.}

\newcommand{\twopdo}[3]{\left\{\begin{array}{ll} #1 & \text{if } #2 \\ #3 & \text{otherwise} \end{array} \right.}

\newcommand{\threepd}[6]{ \left\{ \begin{array}{lll} #1 & \text{if } #2 \\ 
#3 & \text{if } #4 \\ #5 & \text{if } #6 \end{array} \right.}

\newcommand{\threepdo}[5]{ \left\{ \begin{array}{lll} #1 & \text{if } #2 \\ 
#3 & \text{if } #4 \\ #5 & \text{otherwise} \end{array} \right.}

% Bold letters
\newcommand{\X}{\mathbf{X}}
\newcommand{\Y}{\mathbf{Y}}
\newcommand{\Z}{\mathbf{Z}}
\newcommand{\E}{\mathbb{E}}
\newcommand{\I}{\mathbf{I}}
\newcommand{\m}{\mathbf{m}}
\newcommand{\f}{\mathbf{f}}
\newcommand{\e}{\mathbf{e}}
\newcommand{\V}{\mathbf{V}}
\newcommand{\W}{\mathbf{W}}
\newcommand{\R}{\mathbf{R}}
\newcommand{\A}{\mathbf{A}}
\newcommand{\B}{\mathbf{B}}
\newcommand{\D}{\mathbf{D}}
\newcommand{\T}{\mathbf{T}}
\newcommand{\F}{\mathbf{F}}
\renewcommand{\O}{\mathbf{O}}
\renewcommand{\H}{\mathbf{H}}
\newcommand{\g}{\mathbf{g}}
\renewcommand{\r}{\mathbf{r}}
\newcommand{\s}{\mathbf{s}}
\newcommand{\dd}{\mathbf{d}}
\newcommand{\uu}{\mathbf{u}}
\newcommand{\w}{\mathbf{w}}
\newcommand{\x}{\mathbf{x}}
\newcommand{\vv}{\mathbf{v}}
\newcommand{\y}{\mathbf{y}}
\newcommand{\z}{\mathbf{z}}
\newcommand{\Beta}{\boldsymbol{\beta}}
\newcommand{\bveps}{\boldsymbol{\varepsilon}}
\newcommand{\bvphi}{\boldsymbol{\varphi}}
\newcommand{\btheta}{\boldsymbol{\theta}}
\newcommand{\bups}{\boldsymbol{\upsilon}}
\newcommand{\bgamma}{\boldsymbol{\gamma}}
\newcommand{\bpi}{\boldsymbol{\pi}}
\newcommand{\arrowp}{\stackrel{p}{\rightarrow}}
\newcommand{\0}{\mathbf{0}}
\newcommand{\Prob}{\mathbb{P}}
\newcommand{\nn}{\nonumber}
\newcommand{\indf}{\mathds{1}}

% Theorems
\newtheorem{theorem}{Theorem}
\newtheorem{lemma}{Lemma}
\newtheorem{claim}{Claim}
\newtheorem{axiom}[theorem]{Axiom}
\newtheorem{definition}{Definition}
\newtheorem{corollary}{Corollary}
\newtheorem{proposition}{Proposition}
\newtheorem{assumption}{Assumption}
%\newenvironment{proof}[1][Proof]{\textbf{#1.} }{\ \rule{0.5em}{0.5em}}

% Illustration
\theoremstyle{definition}
\newtheorem{example}{Example}%[subsection]
\newtheorem{hypothesis}{Hypothesis}%[subsection]

% Claims in the theorems (without numbering)
\theoremstyle{remark}
\newtheorem*{claimnn}{Claim}

% Title
\title{Incentives for Preventive Healthcare Provision%\thanks{I would like to thank XXXX for their valuable comments. I also thank participants at the MPSA, NYU PE Seminar, and SPSA for their suggestions. All the remaining errors are entirely my responsibility.}
}
\author{
Danilo Freire
\and
Umberto Mignozzetti
}

\date{}

\begin{document}
\maketitle

\singlespacing
\begin{abstract}
  \begin{footnotesize}
    \noindent Viral epidemics are among the leading causes of death in humans and the Zika virus in Brazil warned the world about the dangers of the \textit{Aedes aegypti} mosquito. Since teams of healthcare workers take preventive care efforts, understand how to incentivize teams is essential to improve these interventions. However, most of the literature focuses on individual nurses or teachers, overlooking team efforts, team monitoring, and peer pressure. In this chapter, I bridge this gap by running a field experiment to determine which types of monetary incentives improve teamwork in the Brazilian case. The trial consisted of hired and trained subjects that had to visit houses and help residents to exterminate mosquito breeding sites. I randomly assigned performance monetary rewards in the form of individual and peer performance bonuses. I find that bonuses increased the number of cleaned breeding sites in both treatment status, but team bonuses also improved larvae extermination. Overall, the intervention decreased the disease incidence in 10.3\%, but the results seem not to be robust. This chapter has implications for the design of preventive healthcare interventions.
    
    \noindent \textbf{Keywords:} XXXX.
  \end{footnotesize}
\end{abstract}

\doublespacing

\section{Introduction}

The recent SARS-COV-2 epidemic warns the world about the dangers of viral infections. In 2015, another outbreak in Brazil frightened the globe: the Zika virus epidemic. The Zika virus is a vector-borne disease, with mild effects in adults, but disastrous consequences for fetuses. The outbreak was responsible for a 500-fold increase in Brazil's congenital disabilities, only in 2015. The Zika virus is spread by the \textit{Aedes aegypti} mosquito, which is well known in tropical countries, as it is also the vector for dengue, malaria, and chikungunya. According to the CDC, these diseases together had killed more than 25 thousand people worldwide in 2016.

Despite the use of pesticides and transgenic mosquitoes, specialists agree that preventive healthcare policies are the most sustainable and effective method to fight the \textit{Aedes aegypti} \citep{eisen2009proactive}. The mosquito prevention requires teams of preventive healthcare workers to visit households, helping people to find and exterminate the mosquitoes breeding sites. To this extent, understanding how to improve the productivity of these healthcare workers is crucial for welfare.

One aspect that is often neglected by the medical literature is that better incentives may improve performance. However, the political economy literature has no clear-cut evidence in the case of healthcare workers. On the one hand, economists show that for teachers, increasing monetary compensation, paired with extensive monitoring, improve school attainments \citep{duflo2012incentives}. On the other hand, \citet{Banerjeeetal2008} showed that to incentivize overworking nurses may be ineffective, as they may be performing at their highest level. No paper has to this date investigated how to improve service provision in teams of healthcare bureaucrats. Despite this fact, teams play a significant role in healthcare: a person that goes to a hospital usually interacts with many nurses, doctors, and assistants to diagnose and treat her problem. Teamwork demands cooperative efforts of agents that have to share information and monitor each other's work to ensure an effective outcome. 

In this chapter, I study how to incentivize teams of preventive healthcare workers in Brazil. I build a principal-agent model, along the lines of \citet{itoh1991incentives}, to uncover the main perils in incentivizing these teams. In the model, agents divide effort between doing their job or helping other agents to do their jobs. The principal, a healthcare governmental agency, receives a continuous probabilistic benefit from lower infection rates, which has a higher chance of happening when the agents put more effort. The healthcare agency manipulates the workers' incentives by changing the contracts offered to the workers. The principal-agent model generates three main predictions. In the baseline model, where the healthcare agency gives no individual or collective incentives, the health workers exert zero effort in equilibrium. When the healthcare agency provides individual bonuses, the health worker puts effort toward her work, but there is no effort to help each other. This improves the agency's welfare, at a limited rate, depending on the cost of exerting effort. Finally, when we promote peer incentives, the result depends on how much does it cost for workers to specialize in their jobs. In the case of \textit{Aedes aegypti} prevention, as the task is simple and health workers have no gains of specialization, giving collective bonuses increase the healthcare agency's welfare, as it raises both the subject's own and helping efforts. It should be expected more aggregated outcomes in collective bonus schemes, followed by the individual bonus schemes, and no benefits from no bonuses.

I test this theory by running a field experiment in Rio Verde, a municipality in the Brazilian state of Goias. I hired and trained teams of health workers to perform a simplified preventive healthcare task: visit households to search for potential breeding sites and \textit{Aedes aegypti} larvae. While all participants worked in pairs and have the same duties, they faced three different contractual schemes. In the control group, workers received a flat compensation, untied to their performance. In the individual treatment group, I assigned workers to a contract where workers were ranked based on their accomplishments, and subjects with performance above-median received a bonus. Finally, I assigned pairs to a collective treatment group based on the aggregated performance of the team. Pairs with performance above the median received the rewards.

To measure their performance, I looked into two primary outcomes: the workers' field production collected during the intervention, and the medical reports sent by the municipality to the Ministry of Health disease notification system. For the field outcomes, I find that subjects in the individual and collective treatment groups visited fewer households than subjects in the control group. This could be seen as paradoxical, but as the control participants had no performance incentives, they mostly walked around the neighborhood they were assigned without entering houses. Individual bonus groups eliminated 25.1 breeding sites more than the control groups, while the collective bonus groups eliminated 21.5 breeding sites more than the control group. In terms of larvae extermination, collective bonus groups perform 16.3\% better than the control group, while the individual bonus groups perform similarly to the control group. Finding larvae is the highest yield outcome for preventing \textit{Aedes aegypti}, as larvae are at the final stage of becoming a mosquito.

Using governmental data, I found weak evidence supporting that the intervention lowered the \textit{Aedes aegypti} disease in Rio Verde. Using a differences-in-differences estimator with all three treatments against a pure control of houses that have received no types of interventions, I show that the experiment lowered the disease incidence in 10.3\%. However, the results become insignificant when I consider time windows smaller than 120 days before and after the intervention. There is no effect when I disentangle each treatment individually, but the direction of the results are consistently negative in the collective bonus intervention. The null findings here highlight the difficulty of translating workers' productivity into medical outcomes.

This chapter makes three major contributions to the political economy of preventive healthcare provision literature. First, it adds to a growing scholarship on how incentives affect the efficiency of health care interventions. \citet{Lloydetal1992} and \citet{Toledo2007} show significant effects of community-based field interventions against the \textit{Aedes aegypti}. \citet{Erlanger2008} question these interventions' efficacy vis-a-vis chemical treatments and vaccines. In a recent meta-analysis, \citet{Mignozzettietal2020a} show that community interventions may have a weakly positive effect in terms of \textit{Aedes aegypti} control. In this paper, I highlight that the medical literature may have two shortcomings: first, it lacks a study of the best implementation methods for preventive health care. When the incentives are well designed, the performance in the field increases consistently; and second, the results also imply that performance maps imperfectly into improved entomological parameters, such as the number of larvae per 1000 inhabitants in a given region (Breteau Index) or the number of infections reported before and after the intervention. This can potentially signal that the interventions are ineffective even when the field productivity is high.

Second, this experiment contributes to studies about how to increase the productivity of the public sector. Most of the work about bureaucratic productivity focus on the incentives of a given bureaucrat to improve their outputs. Several papers find that increasing payment enhances the work of teachers and nurses \citep{Banerjeeetal2008, duflo2012incentives, gertler2013using, britton2016teacher, muthmainnah2017improving}. \citet{muthmainnah2017improving} review 25 articles, showing that better salaries improve productivity and the working satisfaction of nurses. \citep{Banerjeeetal2008} show that when the efforts are binding, increasing wages and monitoring of nurses is ineffective. Similarly, \citet{duflo2012incentives} demonstrate that monitor and rewarding teachers is effective in improving education outcomes. This chapter builds upon on their work by studying how to incentivize teams instead of individual workers. Teams are the most common form of working organization, and in the public sector, workers seldom provide services individually. However, different from \citet{bandiera2013team}, this paper does not allow teams to self-select members. My approach simplifies the task of isolating the effect of compensation, eliminating a sorting effect that would happen if we allow workers to choose their peers.

Finally, this research contributes to understanding how to incentivize teams of workers. Incentivizing teams is to tread carefully around the tension between cooperation and conflict. On the one hand, team effort provides a cheap way to monitor free-riding \citep{Holmstrom1982}, and it may increase complementarities of talents and coordination in actions \citep{holmstrom1990regulating, itoh1991incentives}. On the other hand, team workers can collude against the principal \citep{kofman1996optimality}, and within teams, there are also incentives for competition and sabotage \citep{lazear1995personnel}. \citet{bolton2005contract} summarizes the literature and the incentives faced by organizations. Empirically, several papers investigated the output in firms \citep{bandiera2013team}, the peer effects of implementing tasks in the same environment, where workers can witness the other workers' efforts \citep{FalkIchino2006}, and the network effects in collective production \citep{amodio2019inputs, Bursztynetal2014, Castilloetal2014}. I show that collective incentives are adequate to improve field productivity. Qualitative research during the field points that workers find it easier to help each other when their incentives are aligned. The shared risk between team workers makes them more prone to find larvae, which is considerably harder to find than breeding sites. This improves welfare, as larvae are the last stage before a disease-spreading mosquito.

\section{A Theory of Service Provision under Moral Hazard in Teams}

Consider a strategic interaction between two healthcare workers and a governmental agency. The agency hires healthcare workers to provide preventive healthcare. And the workers decide the effort they will exert, conditional on the compensation scheme they receive.

From the perspective of healthcare workers, teams perform preventive healthcare tasks. To inspect a house and eliminate mosquito breeding sites, preventive healthcare workers visit the households in pairs or triplets. This supposedly increases efficiency as it helps them cover more territory and be more careful in the parts they inspect individually. Team structure also helps in terms of lowering the free-riding and increasing the incentives for cooperative work. To capture this, I assume that a given worker $i$ divides her effort into two tasks: own work effort ($e_i$) and helping efforts ($h_i$). To capture the costs of effort, I assume that we have the following convex cost function $\dfrac{c}{2}(e_i^2 + h_i^2)$, where $c>0$ represent the changes in marginal costs of effort and help.

By the nature of the preventive care work, the policy outcome is probabilistic. Preventive healthcare aims to lower the chance of a person gets sick. For example, washing hands lower the chance of getting the flu, but from time to time, a person gets flu regardless of how many times she washes her hands. The only difference is that a person that washes hands constantly has a lower chance of getting sick and will get sick fewer times during her lifespan. For this reason, denote the probability of a successful provision of the service, by a pair $i$ and $j$ of healthcare workers, as $P(e_i, h_i, e_j, h_j)$. For simplicity, I assume that the probabilities are additively separable, with $P(e_i, h_i, e_j, h_j) = P_i(e_i, h_j) + P_j(e_j, h_i)$. To make the output more concrete, assume that a given worker $i$ has an individual probability of $P_i(e_i, h_j) = e_i(1+h_j)$ of successfully lower the disease burden with her effort ($e_i$) and the help of her peer ($h_j$).

From the perspective of the agency, the successful work grants a benefit of $B>0$. This benefit represents the social gains of lowering the disease incidence in the municipality. For example, the pay off could be the amount of money saved hospitalization or the economic benefit of fewer workers missing jobs due to diseases transmitted by the \textit{Aedes aegypti}. The agency incentivizes workers with salaries with three components: a flat salary without bonuses, individual compensation, and a bonus when the entire team achieves their collective objectives. These formulations allow me to disentangle which scheme gives the highest yield in terms of disease prevention.

The optimal scheme to improve welfare resumes into a contract where the governmental agency offers a compensation package, and the two healthcare providers choose effort accordingly. From the perspective of the worker, the expected utility of effort, provided that the compensation depends of the successful provision of the service, and that the benefit from salary is given by a concave function $u(.)$. For concreteness, I use the function $u(x) = \sqrt{x}$. The expected utility of effort is then the compensation for each scenario, times their probability. It is equal to:

\begin{eqnarray*}
    U_i(e_i, h_i; e_j, h_j) & = & P_i(e_i, h_j)P_j(e_j, h_i)u(s+w_{both}) + \\
    & & P_i(e_i, h_j)(1-P_j(e_j, h_i))u(s+w_{one}) + \\
    & & (1-P_i(e_i, h_j))P_j(e_j, h_i)u(s+w_{one}) + \\
    & & (1-P_i(e_i, h_j))(1-P_j(e_j, h_i))u(s) - \dfrac{c}{2}(e_i^2 + h_i^2)
\end{eqnarray*}

And this utility is symmetric for the worker $j$. To simplify things, consider that the compensation under an unsuccessful service provision is equal to zero: $u(s)=0$, or $s=0$. From the perspective of the governmental agency, the benefit from offering the compensation is equal to the social benefit in each provision scenario minus the compensation times the likelihood of the scenario:

\begin{eqnarray*}
    U_A(w_{both}, w_{one}, s; e_i, h_i, e_j, h_j) & = & P_i(e_i, h_j)P_j(e_j, h_i)(2B - 2w_{both}) + \\
    & & P_i(e_i, h_j)(1-P_j(e_j, h_i))(B-2w_{one}) + \\
    & & (1-P_i(e_i, h_j))P_j(e_j, h_i)(B-2w_{one})
\end{eqnarray*}

The problem is a standard contract theory problem, where we have three compensation packages: no bonus ($w_{both} = w_{one} = 0$), individual bonus ($w_{both} = w_{one} \neq 0$), and collective bonus ($w_{both} \neq w_{one}$, $w_{both} \neq 0$, and $w_{one} \neq 0$). The optimal contract is then the solution for the following constrained optimization problem:

\begin{eqnarray*}
    \max_{w_{both}, w_{one}} U_A(w_{both}, w_{one}; e_i, h_i, e_j, h_j) \text{, subject to:} \\
    & (e_i, h_i) \in \arg \max U_i(e_i, h_i; e_j, h_j, w_{both}, w_{one}) \\
    & (e_j, h_j) \in \arg \max U_j(e_j, h_j; e_i, h_i, w_{both}, w_{one})
\end{eqnarray*}

And by choosing an optimal compensation scheme, the agency can induce more effort and more cooperation among workers.

\subsection{Baseline: no incentive bonuses}

When the agency only pays a flat salary without performance incentives, the agency sets the performance compensation equals to zero: $w_{both} = w_{one} = 0$. This changes the expected utility of the agent into the following:

\begin{eqnarray*}
U_i(e_i, h_i; e_j, h_j) & = & - \dfrac{c}{2}(e_i^2 + h_i^2)
\end{eqnarray*}

In this setting, exerting effort generates only costs and no benefits. The agents exert zero effort and help in equilibrium: $e_i^* = e_j^* = 0$ and $h_i^* = h_j^* = 0$. The agency pays the minimum wage accepted by law, and in this chapter, I assume that this wage is equal to zero. Accordingly, the agency expected benefit is equal to:

$$ R_A(\text{no bonus}) \ = \ 0$$

The prediction here is that workers will exert no effort, and the agency will have only costs without accruing any benefit. This is the worst-case scenario in terms of the agency's welfare. In the experiment, it matches with the control condition.\footnote{Throughout the paper, I assume that the optimal output for the agency is equal to the welfare-maximizing for the society. Although this is not always the case, it is a reasonable simplification for the chapter purposes. Adding a difference in here would only complicate my results, providing no further insights into designing an incentive scheme to maximize welfare.}

\subsection{Performance bonus for individuals}

When the benefits are tied to the individual performance, the governmental agency compensates the workers for their effort, without tying their bonuses with helping other workers. This means that $h_i^* = h_j^* = 0$, but the optimal level of effort is always positive. Let $\alpha$ the bonus that the governmental agency offers to the workers conditional on lowering the disease incidence. Each work then evaluates the utility of the bonus as $u_i(\alpha) = \sqrt{\alpha}$, as the utility of the agent is concave, representing the risk aversion of the agent. The optimization problem, from the perspective of the agent, is to maximize her expected utility, given the bonus offered by the governmental agency upon achieving a successful provision:

\begin{eqnarray*}
\max_{e_i} U_i(e_i) & = & \max_{e_i} \left[P_i(e_i, 0)\sqrt{\alpha} - \dfrac{c}{2}(e_i^2)\right] \\
 & = & \max_{e_i} \left[e_i\sqrt{\alpha} - \dfrac{c}{2}(e_i^2)\right]
\end{eqnarray*}

Both workers face the same incentive structure, so the equilibrium is symmetric, with $e_i = e_j$. Taking the derivative, and equating it to zero, gives us the following optimal effort level: $e_i^* = e_j^* = \dfrac{\sqrt{\alpha}}{c}$. The optimal level of effort represents when the marginal benefit from the governmental agency compensation is equal to the marginal cost of effort. The benefit of a successful provision of the healthcare services is equal to, for both players, the total benefit times the chance of a successful provision, times the difference between the benefit accrued from a successful provision minus the bonus paid for the worker. To compute the optimal contract that the governmental agency have to offer in this setting, we have to choose $\alpha$ to maximize the agency's expected utility:

\begin{eqnarray*}
\max_{\alpha} R_A(e_i^*(\alpha), e_j^*(\alpha)) & = & \max_{\alpha} \left[2 \dfrac{\sqrt{\alpha}}{c} (B-\alpha)\right]
\end{eqnarray*}

The optimal bonus is equal to $\alpha^* = \dfrac{B}{3}$. Plugging the optimal bonus and the optimal effort levels in the welfare function, we find the agency's expected revenue, which in this case is equal to:

$$ R_A(\text{individual bonus}) \ = \ \dfrac{4}{c} \left[ \dfrac{B}{3} \right]^\dfrac{3}{2} $$

The revenue is always greater than zero, provided that the costs of investing effort and the social benefits of the policy are greater than zero.

\subsection{Performance bonus for teams}

When the governmental agency sets rewards for both the individual and the peer performance, healthcare workers divide their time between effort in their job ($e_i$) and help peers get their job done ($h_i$). Helping may take many forms: from simple peer pressure and monitoring to ensure the peer is doing the job well to divide tasks and exploit complementarities in their duties efficiently. From the governmental agency's perspective, incentivizing workers to help each other can potentiate the chances of getting a successful service provision. From the standpoint of the healthcare workers, the optimal effort becomes a mix of her own efforts and the helping efforts that maximize her expected utility, provided that both efforts are incentive-compatible. Incentive compatible means, in the case of healthcare workers, that both efforts are exerted to match the governmental agency's contractual incentives. The expected utility for the agent $i$ consists of the chance of getting bonus times the utility of the bonus. This iterated in each of the possible service provision scenarios.

\begin{eqnarray*}
\max_{(e_i,h_i)} U_i(e_i, h_i; e_j, h_j) & = & \max_{(e_i,h_i)} \{e_j(1+h_i)e_i(1+h_j)u(w_{ij}) + \\
& & e_j(1+h_i)(1-e_i(1+h_j))u(w_{j}) + \\
& & e_i(1+h_j)(1-e_j(1+h_i))u(w_{i}) - c\dfrac{e_i^2+h_i^2}{2}\}
\end{eqnarray*}

Again, both players are ex-ante similar in terms of their expected utilities. This allows me to solve for the Symmetric Nash Equilibrium, where:

\begin{enumerate}
    \item The effort of a given healthcare provider ($i$) is equal to the effort of the other healthcare provider ($j$): $e_i = e_j = e$.
    \item The help of one healthcare provider is equal to the help of the other healthcare provider $h_i = h_j = h$.
    \item Both agents have the same utility function, $U_i(.) = U_j(.)$.
\end{enumerate}

Maximizing the expected utility leads us to the optimal effort and help for the healthcare provider, given the contractual scheme. The optimal provision have to satisfy the following equation:

\begin{eqnarray*}
\dfrac{\partial U}{\partial e} & = & e(1+h)^2(u(w_{ij})-u(w_{i})-u(w_{j})) + (1+h)u(w_{i})-ce \\
\dfrac{\partial U}{\partial h} & = &
e^2(1+h)(u(w_{ij})-u(w_{i})-u(w_{j})) + eu(w_{j})-ch
\end{eqnarray*}

To simplify our computation, we consider contracts that vary in one dimension $\alpha$. The individual part of the contractual benefit is equal to multiples of this contract for each situation: own successful provision and the successful colleague provision.

In this example, we restrict our attention to contracts where $u(w_{i}) = u(w_{j}) = \sqrt{\alpha}$ and $u(w_{ij}) = 2u(w_{i})$. These contracts match better the conditions I set in the experiment: they represent the collective gain when recompensing the agent for peer and individual successes. The optimal effort and help becomes $e_i^* = e_j^* = \dfrac{\sqrt{c\alpha}}{c-\alpha} $ and $h_i^* = h_j^* = \dfrac{\alpha}{c-\alpha}$. The governmental agency now maximizes the bonus for the individual provision ($\alpha$), given that the collective provision is higher than it. In the appendix, we solve for contracts that vary in both dimensions. The expected revenue of the agency is equal to the chance of lowering the disease burden times the benefit minus costs of pay the healthcare provider's salaries:

\begin{eqnarray*}
\max_{\alpha} R_A(e_i^*(\alpha), h_i^*(\alpha), e_j^*(\alpha), h_i^*(\alpha)) & = & \max_{\alpha} \left[2 \dfrac{c\sqrt{c\alpha}}{(c-\alpha)^2} (B-2\alpha)\right]
\end{eqnarray*}

The derivative of this equation is the polynomial in the second degree in $\alpha$. By the intermediate value theorem, it is easy to see that it has one solution for $\alpha \in [0, B]$. To illustrate, consider the Figure \ref{figrevmax}, where I plot the expected revenue for the agency ($R_A(\alpha)$), as a function of the offered compensation $\alpha$, and the three types of contract: no productivity incentives, individual productivity incentives, and peer productivity incentives, fixing $B = c = 1$.

\begin{figure}
\centering
\includegraphics[width=1\textwidth]{figures/figrev.pdf}
\caption{Governmental Agency's Revenue in terms of Contract and Compensation}
\label{figrevmax}
\end{figure}

Therefore, the maximum benefit that the agency can achieve is higher when the agency compensates the workers for their cooperation. When the agency incentivizes the team effort, workers can on margin do better by sharing the benefit of a successful provision.\footnote{The limits for this reasoning appear when there are gains of specialization. For instance, if the costs for effort and help were such that $\dfrac{c(e+h)^2}{2}$, then diverting energy for help is costly, on margin at the same rate as the given effort provision. In this case, the compensation to stimulate cooperation has to be significantly higher, and even in this case there will be incentives for not cooperate.}

\section{Experimental Design}

In this section, I present the experimental design. I explain the intervention stages: the municipality selection, and the recruitment and training of the subjects. Then, I discuss the intervention, explaining the treatment delivery and the randomization. Finally,  I discuss the performance measurements and the municipality disease information. Although I focus only on the main details here, the Supplementary Materials in Appendix A discusses all the data in depth.

\subsection{Pre-treatment}

There were three crucial steps taken in the pre-intervention stage: municipality selection, recruitment and training, and the randomization.

In terms of municipality selection, I ran the experiment in the municipality of Rio Verde, in the State of Goias. Rio Verde is the fourth-largest municipality in the state, comprised of 229 thousand inhabitants. Its income comes chiefly from soybean exports and other agricultural produce.

In 2018, the municipality experienced a substantial increase in dengue fever cases. Among the towns I had an agreement to run the experiment, Rio Verde experienced the most substantial disease occurrence.\footnote{I had the authorization to run the experiment in Bauru and Ribeirao Preto, both in the Sao Paulo State. However, these municipalities had a very low case incidence from January to March 2018: 8 and 142.}

I used a Facebook ad to recruit potential participants in the municipality. The ad redirected participants to a Google Form that requested information about the participant and an e-mail address. Using the e-mail address, I send a Qualtrics survey with pre-treatment demographic questions. Upon completion, I sent them the date and timing for the training session held on May 4, 2018.

During the training session, my research assistants and I presented the \textit{Aedes aegypti} vector's biology and how it transmits the diseases. The meeting also taught participants how to find and exterminate the mosquito breeding sites. Participants then had to read and sign the informed consent to participate in the research.

The intervention happened the following day (May 5). After the final training meeting, I performed the randomization and directed participants to their working headquarters based on the randomization status. There was around 50 percent of attrition from the registration to the main intervention participants. Table \ref{tabcovbal1}, Panel A displays the attrition tests, Panel B presents the balance tests for the participants' demographics, and Panels C and D the territorial assignment before and after the matching on census-sector characteristics.

% Table created by stargazer v.5.2.2 by Marek Hlavac, Harvard University. E-mail: hlavac at fas.harvard.edu
% Date and time: Sat, Feb 01, 2020 - 17:56:00
\begin{table}[!htbp] \centering 
  \caption{Covariate Balance on Individual Characteristics} 
  \label{tabcovbal1} 
\begin{tabular}{@{\extracolsep{5pt}} lcccccc} 
\\[-1.8ex]\hline 
\hline \\[-1.8ex] 
Variable & Attrition & Control & Individual & Collective & F-Statistic & P-Value \\
\hline \\[-1.8ex] 
& & \multicolumn{5}{l}{Panel A: Attrition pattern test} \\
\hline \\[-1.8ex]
Age &  26.17 & 25.04 & 26.07 & 26.00 & 0.461 & 0.710 \\ 
Female &   0.72 &  0.79 &  0.79 &  0.74 & 0.805 & 0.492 \\ 
Has Vehicle &   0.35 &  0.27 &  0.37 &  0.37 & 0.651 & 0.582 \\ 
N & 341 & 73 & 71 & 73 &  &  \\ 
\hline \\[-1.8ex] 
\end{tabular} 
\begin{tabular}{@{\extracolsep{5pt}} lccccc} 
Variable & Control & Individual & Collective & F-Statistic & P-Value \\
\hline \\[-1.8ex] 
& \multicolumn{5}{l}{Panel B: Covariate balance on individual characteristics} \\
\hline \\[-1.8ex]
Altruist &  0.63 &  0.68 &  0.62 & 0.239 & 0.788 \\ 
Age & 25.01 & 26.00 & 26.90 & 0.754 & 0.472 \\ 
Religiosity &  0.53 &  0.38 &  0.45 & 1.487 & 0.228 \\ 
Pol. Engagement &  0.69 &  0.65 &  0.68 & 0.163 & 0.849 \\ 
Soc. Engagement &  0.59 &  0.62 &  0.67 & 0.454 & 0.636 \\ 
Above 2 Min. Wage &  0.34 &  0.35 &  0.29 & 0.336 & 0.715 \\ 
FB Popularity &  0.53 &  0.56 &  0.58 & 0.175 & 0.840 \\ 
\hline \\[-1.8ex] 
& \multicolumn{5}{l}{Panel C: Unmatched Covariate balance on territorial assignment} \\
\hline \\[-1.8ex] 
Number of Houses & 276.41 & 243.58 & 255.20 &  2.368 & 0.098 \\ 
Number of Households & 885.67 & 750.34 & 785.18 &  3.765 & 0.026 \\ 
Avg. Household Size &   3.21 &   3.07 &   3.07 &  6.566 & 0.002 \\ 
Log Avg. Income &   6.47 &   6.86 &   6.85 &  9.213 & 0.000 \\ 
Cases Before Treatment &  13.78 &   5.94 &   4.66 & 15.638 & 0.000 \\ 
\hline \\[-1.8ex] 
& \multicolumn{5}{l}{Panel D: Matched covariate balance on territorial assignment} \\
\hline \\[-1.8ex] 
Number of Houses & 276.41 & 282.04 & 281.19 & 0.068 & 0.935 \\ 
Number of Households & 885.67 & 903.04 & 900.39 & 0.060 & 0.942 \\ 
Avg. Household Size &   3.21 &   3.21 &   3.20 & 0.008 & 0.992 \\ 
Log Avg. Income &   6.47 &   6.45 &   6.45 & 0.084 & 0.920 \\ 
Cases Before Treatment &  13.78 &  10.24 &  10.77 & 1.443 & 0.240 \\ 
\hline
Pairs & 34 & 34 & 35 &  &  \\ 
N & 68 & 68 & 70 &  &  \\
\hline\hline \\[-1.8ex] 
\end{tabular}
\end{table}

There were no distinct patterns on the attrition, reinforcing that it was at random. The participants' demographics are similar across the treatment status, and the matching successfully balanced the territorial assignment.

\subsection{Intervention}

I assigned participants to three types of monetary incentives: a no-performance bonus (control group), an individual performance bonus (individual treatment), and a peer performance bonus (collective treatment).

Before the primary intervention, I run a pilot three weeks earlier with twenty participants to measure the applicability of the experiment, and to calibrate the data collection. In the pilot, I find that participants could visit around sixty households during the intervention day. This is because about one-third of houses were generally empty, and the yards in the municipality are usually small, facilitating the work.

The intervention started around 11:00 AM and ended around 6:00 PM on May 5, 2018. To avoid spillover, I divided the participants into three different headquarters, away from each other, and each was applying a different treatment. The intervention was delivered around 10:30 AM after participants had received their monitoring cellphones and their working kits. The head research assistant told the participants collectively the information of each treatment status. In a one-on-one discussion, the research assistants reinforced the treatment for each of the pairs. The treatment delivery follows below:

\begin{itemize}
    \item In the control group, we told participants that we would measure their performances in the field. We also told them that the bonus would be assigned without a performance assessment.
    \item In the individual treatment group, we told participants that we would measure their performances in the field. We explained that we would rank the individual performances, and anyone reaching performance above the median would double their compensation.
    \item In the collective treatment group, we told participants that we would measure their performances in the field. I explained that I would rank the team performance, which was the sum of both participant's performances, and teams reaching performance above the median would double their compensation.
\end{itemize}

The field compensation was BRL 110.00. The participants that successfully doubled it received BRL 220.00. Each participant within a headquarter was paired randomly with another participant. The pairs received a leaflet with a map of the area they should cover. The route was comprised of two to three blocks, containing around 120 houses. The blocks were within a census sector, to facilitate the use of census data afterward.

A typical visit to a household consisted of the following steps. First, the participants' ring the doorbell in the house and explained that the municipality was experiencing a dengue fever outbreak.\footnote{Despite the widespread news coverage of Zika in the US, people have little knowledge about it in Brazil. Dengue, on the other hand, has news coverage in Brazil since the 1980s. In Brazil, the \textit{Aedes aegypti} is referred by households as the Dengue mosquito.} They instructed dwellers on how to lower the disease incidence, protect their families, and handed in a leaflet with information about it.

Then, they requested households to enter their houses and inspect their yards. When granted permission, they had to check the yard to find clean breeding sites and exterminate larvae. Upon cleaning a breeding site, they had to report it with their cellphones by taking a picture. When they discovered larvae, they had to record a video with the larvae before exterminating it. At the end of their visit, they had to take a picture of the household, to account for their presence in the place.

\subsection{Post-treatment}

After the treatment takes place, I computed the performance in the field by counting the number of houses visited, the number of breeding sites discovered and eliminated, and the number of larvae exterminated. The information comes from the cellphones used by participants in the intervention. I had research assistants cleaning the pictures to ensure that all the images reported houses or breeding sites and that all videos were from larvae. Based on the photos and videos, I computed the performance for each participant and paid the bonus according to the treatment status assigned to the person.

In the months that followed the intervention, I requested the mayor's office the reports on the disease incidence of dengue fever, from January to September 2018. In this dataset, I have the hospitalization of every person in the municipality that had Dengue fever during the period. Using the dataset, I georeferenced all the cases in the city. Moreover, as this is sensitive healthcare information, protected by the Brazilian law, they agreed to provide it only one year after the intervention. I stored the data safely and encrypted, as it has addresses, full names, social security numbers (Brazilian CPFs), and other sensitive personal information.

\section{Results}

I measure two performance outcomes: field productivity and disease incidence. The field productivity consisted of the cellphone data collected by the subjects during the intervention. The disease incidence comprised data collected by the mayors' office with hospital treatment of people that presented symptoms of dengue fever from January to September 2018. Below, I discuss how the experiment affected each of these outcomes.

\subsection{Field productivity}

For the field productivity, I have four indicators. First, the number of houses visited. Second, the number of homes visited in less than two minutes. Third, the number of breeding sites removed and cleaned. Finally, the number of larvae exterminated. The results follow in Table \ref{prodregs}.

\begin{table}[!htbp] \centering 
  \caption{Field Productivity} 
  \label{prodregs} 
\begin{tabular}{@{\extracolsep{5pt}}lcccc} 
\\[-1.8ex]\hline 
\hline \\[-1.8ex] 
 & \multicolumn{4}{c}{\textit{Dependent variable:}} \\ 
\cline{2-5} 
\\[-1.8ex] & Houses & Houses & Breeding & Larvae \\ 
& Visited & Visited ($<$ 2min.) & Sites & Exterminated \\ 
\hline \\[-1.8ex] 
 Individual Bonus & $-$9.879$^{**}$ & $-$8.044$^{**}$ & 25.118$^{***}$ & 0.022 \\ 
  & (4.906) & (4.018) & (5.251) & (0.058) \\ 
  & & & & \\ 
 Collective Bonus & $-$7.554$^{*}$ & $-$6.479$^{*}$ & 21.512$^{***}$ & 0.163$^{**}$ \\ 
  & (4.428) & (3.571) & (4.049) & (0.076) \\ 
  & & & & \\ 
\hline \\[-1.8ex] 
Observations & 197 & 197 & 197 & 197 \\ 
Residual SE & 27.301 & 22.153 & 30.545 & 0.425 \\ 
\hline 
\hline \\[-1.8ex] 
\textit{Note:}  & \multicolumn{4}{r}{Robust standard errors in parentheses. $^{*}$p$<$0.1; $^{**}$p$<$0.05; $^{***}$p$<$0.01} \\ 
\end{tabular} 
\end{table}

Incentivized workers visit fewer households than non-incentivized workers. This result may be counter-intuitive, but the performance incentives reward more the participants that exterminate larvae and cleaned breeding sites, instead of just visiting households. In the control group, they had no incentives to do a thorough job, but they were still instructed to cover all the territory to receive compensation. Non-incentivized teams had only the incentives to take pictures of houses, regardless of their actual commitment to visit a home and exterminate breeding sites and larvae. For instance, the number of household pictures taken two minutes apart from each other was -8.04 and -6.48 lower in the incentivized individual and collective bonus groups than in the control group. Visit a household in two minutes means that, either the house was empty, or that the pair decided not to visit the home. In both cases, the house yard was not inspected.

However, incentivized teams find and clean more breeding sites than non-incentivized ones. On average, the individually incentivized workers find 25.1 and 21.5 more breeding sites than the non-incentivized control group. The workers in the collective bonus treatment cleaned on average 50\% more breeding sites than the control group. Breeding sites are relatively easy to find and require just a careful work in the yard. Almost any recipient that can hold seated clean water can be a breeding site. Moreover, many people leave plant plates filled with water, and simple instructions to fill up the plate with sand can help terminate a breeding site.

Collective incentivized teams are more effective in finding larvae. In our study, collectively incentivized subjects have 16.3\% more chance to find larvae than non incentivized workers. Finding larvae is harder, and the detection chances profit from teamwork. If teams are cooperating, searching the house becomes more natural, and they can coordinate to do a more thorough job. Additionally, the detection tends to be excludable: one individual finding larvae makes the chances of keep finding larvae in the household considerably smaller. Collective bonus is not excludable within the teams, making them cooperate more to get the reward.

Monetary incentives work. Incentivized workers visit fewer households, but do a considerably better job in the houses they go to. Also, the excludability of the benefits seems to make individually incentivized workers more prone to search for breeding sites. Non-excludability of the bonus makes workers more inclined to search for larvae, which are considerably harder to find. They still search for breeding sites in the process but focusing more on finding larvae, as it was more rewarding.

\subsection{Disease Incidence Outcomes}

To evaluate the effect of the treatment on the disease incidence, I run two models. The first model is a differences-in-differences model to assess whether the intervention per se had any effect. The second model compares the disease incidence in each of the treatment statuses with the disease incidence in the control group. In both of the models, the unit of analysis is the census sectors.

In the first model, I look into the disease outcomes in a given time-window, using a differences-in-differences approach. The treatment status represents any of the interventions against a pure-control of households not visited. I chose four windows around the treatment: full data, which comprises 16 weeks around the treatment; 12 weeks around the treatment; eight weeks around the treatment; and four weeks around the treatment.\footnote{The treatment was carried out on the last day of the 17th week of 2018.} Also, to evaluate the differences between individual and collective bonuses, in the last column, I ran a differences-in-differences with the collective treatment as the main intervention, and the individual treatment the control. Table \ref{difindifdiseases} displays the results.

\begin{table}[!htbp] \centering 
  \caption{Disease Incidence -- Differences in Differences Model} 
  \label{difindifdiseases} 
\begin{tabular}{@{\extracolsep{5pt}}lccccc} 
\\[-1.8ex]\hline 
\hline \\[-1.8ex] 
 & \multicolumn{5}{c}{\textit{Dependent variable:}} \\ 
\cline{2-6} 
\\[-1.8ex] & N Infected & N Infected & N Infected & N Infected & N Infected \\
& (full) & (12 weeks) & (8 weeks) & (4 weeks) & (indv. x collect.) \\ 
\hline \\[-1.8ex] 
 DID & $-$0.103$^{*}$ & $-$0.130 & $-$0.167 & $-$0.097 & $-$0.029 \\ 
  & (0.053) & (0.091) & (0.118) & (0.166) & (0.035) \\ 
  & & & & & \\ 
\hline \\[-1.8ex] 
Observations & 7,452 & 3,312 & 1,656 & 828 & 2,201 \\ 
Residual SE & 1.004 & 1.319 & 1.442 & 1.513 & 0.381 \\ 
\hline 
\hline \\[-1.8ex] 
\textit{Note:} & \multicolumn{5}{r}{DID stands for the Differences-in-Differences estimator.}
& \multicolumn{5}{r}{Robust standard error, clustered at the Census-Sector level, in parenthesis.} \\ 
& \multicolumn{5}{r}{$^{*}$p$<$0.1; $^{**}$p$<$0.05; $^{***}$p$<$0.01} \\ 
\end{tabular} 
\end{table}

When comparing any treatment with no treatment, the experiment lowered the incidence of diseases in 10.3\%. The result is consistent around fewer weeks before and after the intervention, but it is not significant. This suggests that the effect is small and weak. Figure \ref{figc360daysdiseaseinc} illustrates the differences in disease incidence using a 60 days window before and after the intervention. The Figure shows a consistent reduction in the disease incidence, which is compatible with the 10.3\% reduction estimated in the differences-in-differences model.

\begin{figure}
\includegraphics[width=\textwidth]{figures/c3diseaseincidence60dw.pdf}
\caption[Disease incidence.]{Disease incidence 60 days around the treatment.\label{figc360daysdiseaseinc}}
\end{figure}

The differences in differences comparing the two primary treatment statuses show an impact of 2.9\% fewer infections in the collective bonus group. However, the result is not significant at standard significance levels.

In the second model, I look into the number of hospitalized households in each of the census sectors, based on the treatment status of pairs working in the census sector. Table \ref{regdiseaseinc1} shows the hospitalizations before the intervention, and then 15, 30, 60, 90, and between 15 to 30 days after the treatment.

\begin{table}[!htbp]
  \caption{Disease incidence in Rio Verde census sectors} \centering 
  \label{regdiseaseinc1} 
\begin{tabular}{@{\extracolsep{5pt}}lcccccc}
\\[-1.8ex]\hline 
\hline \\[-1.8ex] 
 & \multicolumn{6}{c}{\textit{Dependent variable:}} \\ 
\cline{2-7} 
\\[-1.8ex] & Before & 15 days & 30 days & 60 days & 90 days & 30-60 days \\ 
& Intv. & After Intv. & After Intv. & After Intv. & After Intv. & After Intv. \\
\hline \\[-1.8ex] 
 Individual & $-$3.534 & 0.116 & 0.551 & 1.551 & 1.534 & 0.936 \\ 
  & (2.919) & (1.338) & (2.193) & (3.260) & (3.258) & (1.188) \\ 
  & & & & & & \\ 
 Collective & $-$3.008 & $-$1.329 & $-$0.826 & 0.344 & 1.091 & 1.068 \\ 
  & (4.174) & (1.440) & (1.971) & (3.233) & (3.610) & (1.582) \\ 
  & & & & & & \\ 
\hline \\[-1.8ex] 
Observations & 139 & 139 & 139 & 139 & 139 & 139 \\ 
Residual Std. Error & 9.442 & 4.076 & 6.095 & 8.896 & 9.341 & 3.536 \\ 
\hline 
\hline \\[-1.8ex] 
\textit{Note:}  & \multicolumn{6}{r}{Robust Standard Error in Parenthesis. $^{*}$p$<$0.1; $^{**}$p$<$0.05; $^{***}$p$<$0.01} \\
\end{tabular} 
\end{table} 

Compared with the control group, no treatment had a practical influence on the disease incidence in the Rio Verde Census sectors. However, the coefficients tend to be negative, which suggests that should an impact have occurred, the intervention would have probably had lowered the disease incidence.

Why does increased productivity not causes fewer hospitalizations? First, the disease epidemiology suggests that finding a breeding site does not directly imply that all the larvae were exterminated. Although it becomes harder to find, and in our sample, no worker (or pair) found two larvae in the same house, finding a larvae container guarantee that there was a mosquito in a 70 meters radius. The mosquito could have laid eggs anywhere around this perimeter, but it becomes harder to detect. Second, the municipality had a significant incidence of the disease, reaching 3,411 reported infections from January to September. It can be that any efforts in this period only generate a consistent effect when it is above some threshold level. For example, a disease such as measles only become eradicated by herd immunity when the vaccination reaches around 95\% of the population \citep{fox1983herd}. For having the same effect, it could be that I had to assemble a team to visit around 90\% or more of the households. Lastly, our intervention could have affected hospitalizations, increasing hospital admissions around the treated area: remember that the first step in visiting a house was to explain they should go to a doctor upon having any of the symptoms. This is a competing effect, as the number of infections decreases because of the intervention, but the reports for the hospital increase, offsetting the impact.

\section{Discussion}

This paper shows that performance monetary incentives increase the productivity of community healthcare workers in Brazil. Studying the context of \textit{Aedes aegypti} prevention, I contrast three types of contractual incentives: no performance rewards (control group), individual bonus for the most productive workers in the field (individual treatment), and peer reward, to the more productive pairs in the intervention (collective treatment). I find that collective incentives perform better when the job is harder and needs cooperation. Finding mosquito larvae is difficult, and detection by one person virtually excludes the other from the benefit. When the bonus is shared, the pairs collaborate more, and this makes finding larvae easier. On the other hand, individual incentives are better to improve the detection of more accessible but essential to work on, such as cleaning breeding sites. Finally, the control group presented a high incidence of houses visited in less than two minutes apart from each other, which suggests that the workers may cheat without incentives and not perform their jobs accordingly.

The findings here are significant for policy-making. First, I show that performance bonus work also in public healthcare. Although it is known for education \citep{Dufloetal2012} and collective workers in the private sector \citep{Ashrafetal2014a, Ashrafetal2014c}, this is one of the first attempts to study peer incentives for workers performing a public service job. Second, the collective bonus outperforms individual performance rewards when the task is more laborious and excludable. The excludability of the benefit and the difficulty of the task makes cooperation among team workers beneficial to accomplish the job. Finally, as the epidemiology of vector-borne diseases is involved, to enhance productivity to translate into significant effect, I would need a vast pool of workers in the same way as herd immunity works. This teaches us that improve public service provision is challenging, even when the workers perform their tasks effectively.

\clearpage

% Bibliography
\singlespacing
\begin{footnotesize}
\bibliographystyle{chicago}
\bibliography{biblio.bib}
\end{footnotesize}

\clearpage

\appendix

\section{Appendix Content}

In the appendix, I present the APSA Experimental Session Standard Report. This report has all the details on the experiment, and also has additional field information and pictures. In the report, there is the following information:

\begin{enumerate}
    \item Hypothesis
    \begin{enumerate}
        \item Question addressed by the experiment
        \item Hypothesis tested
    \end{enumerate}
    \item Subjects and Context
    \begin{enumerate}
        \item Eligibility and Exclusion Criteria
        \item Recruitment Information
        \item Training Information
    \end{enumerate}
    \item Allocation methods
    \begin{enumerate}
        \item Randomization procedures and details
        \item Pre-treatment balance checks
        \item Details on random assignment and treatment administration
    \end{enumerate}
    \item Treatment descriptions
    \item Results
    \begin{enumerate}
        \item Analysis of outcomes
        \item CONSORT
    \end{enumerate}
    \item Other information
    \begin{enumerate}
        \item NYU IRB
        \item Funding
        \item Pre-registry
        \item Field research assistants thanks note
        \item A few pictures of the field
    \end{enumerate}
\end{enumerate}

\end{document}
